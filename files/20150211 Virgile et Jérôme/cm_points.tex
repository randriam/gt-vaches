\documentclass{article}
\usepackage[utf8]{inputenc}
\usepackage[T1]{fontenc}
% \usepackage[francais]{babel}
\usepackage{math}
\usepackage{unicode}
\usepackage[margin=20mm]{geometry}
\let\ro\mathcal
\let\fr\mathfrak
\DeclareMathOperator\PSL{PSL}
\DeclareMathOperator\PGL{PGL}
\DeclareMathOperator\CM{CM}
\DeclareMathOperator\Imag{Im}
\def\qalg#1#2{\pa{\frac{#1}{#2}}}
\def\F{\mathbb{F}}

\begin{document}
\title{CM points of Shimura curves}

\section{Preliminaries: fixed points of homographies}
\def\arraystretch{.6}

We note that $\PSL_2(ℝ) = \PGL_2^+(ℝ)$, since any real matrix with
determinant~$> 0$ is homothetic to a unique matrix with determinant~$1$.
The group~$\PSL_2(ℝ)$ acts on~$ℙ^1(ℂ)$ by homographies:
$\mat{a&b\\c&d} · z = \frac{az+b}{cz+d}$.
Moreover, since $γ$~is real, we have~$\overline{γ · z} = γ ·
\overline{z}$. This means that it is enough to look at the action
of~$\PSL_2(ℝ)$ on the quotient of~$ℙ^1(ℂ)$ by complex conjugation,
which is~$\ro H ∪ ℝ ∪ \acco {∞}$.


The fixed points for the homographic action of~$γ$ correspond to
(complex) eigenspaces of~$γ$.

\begin{prop}\label{prop:fixed-points}
Two matrices~$γ, γ' ∈ \PGL_2(ℝ)$
have the same fixed points in~$ℙ^1(ℂ)$ iff $ℝ[γ] = ℝ[γ']$.
\end{prop}

This means that a quadratic field~$K ⊂ ℝ^{2×2}$ is determined by its
fixed points in~$\ro H ∪ ℝ ∪ \acco{∞}$. The field is imaginary iff it has
one fixed point in~$\ro H$ and real iff it has two in~$ℝ ∪ ∞$.

Numerically, the matrix~$\mat{a & b \\ c & d}$ with eigenvalues~$λ, λ' =
a+d-λ$ corresponds to the fixed points~$\frac{λ-d}{c}$, $\frac{a-λ}{c}$.

\begin{df}
We say that an element~$γ$ of~$\PSL_2(ℝ)$ is
\begin{enumerate}
\item \emph{elliptic} if it has two complex conjugate fixed points;
\item \emph{hyperbolic} if it has two distinct fixed points in~$ℝ ∪
\acco{∞}$;
\item \emph{parabolic} if it has one single, real fixed point.
\end{enumerate}
\end{df}

Since $\det γ = 1$, it is easy to see that $γ$~is hyperbolic iff
$\abs{\Tr γ} > 2$ (or its discriminant is < 0), elliptic iff $\abs{\Tr γ}
< 2$ (or its discriminant is > 0), and parabolic iff
$\abs{\Tr γ} = 2$ (or its discriminant is 0).

This means that, if $γ$~is algebraic over~$ℚ$, then the algebra~$ℚ(γ)$ is
an imaginary quadratic field if $γ$~is hyperbolic, a real quadratic field
(or $ℚ × ℚ$) if $γ$~is elliptic, and a local $ℚ$-algebra if $γ$~is
parabolic.

Let~$Γ ⊂ \mathrm{SL}_2(ℝ)$ be a discrete subgroup.
A point of~$\ro H/Γ$ is called \emph{elliptic} if it is fixed
by an elliptic element~$γ ∈ Γ$.
\begin{prop}\label{prop:elliptic-cyclic}
Let~$z ∈ \ro H/Γ$ be an elliptic point.
Then the stabilizer~$Γ_z$ of~$z$ in~$Γ$ is a finite cyclic group.
\end{prop}

\begin{proof}
Let~$g ∈ \mathrm{SL}_2(ℝ)$ such that~$g · i = z$.
Then $g^{1} Γ_z g$~fixes~$i$,
and hence included in the stabilizer of~$i$ in~$\mathrm{SL}_2(ℝ)$.
This stabilizer is the group~$\mathrm{SO}_1(ℝ) ≃ ℝ/2πℤ$.
Any discrete subgroup of this compact group is finite and cyclic.
\end{proof}


\begin{prop}\label{prop:elliptic-23}
% Let~$γ ∈ ℤ^{2×2}$ such that~$γ^n = 1$.
Let~$γ ∈ ℝ^{2 × 2}$ be entire over~$ℤ$ and or finite order.
Then the order of~$γ$ is either~$2$, $3$, $4$, or~$6$.
(The order of~$γ$ in~$\PGL_2(ℤ)$ is either~$2$ or~$3$).
\end{prop}

\begin{proof}
Both eigenvalues of~$γ$ are entire over~$ℤ$ and the norm is~$± 1$,
so that the eigenvalues are~$± e^{± i θ}$ for some~$θ ∈ ℝ$.
This implies that~$\mathrm{Tr} θ = 2 \cos θ$.
Since this is also an integer,
the only possibilities for the characteristic polynomial of~$γ$ are
$x^2±1$, $x^2±x±1$, and~$(x±1)^2$.
\end{proof}

\section{Quaternions and complex-multiplication points}%<<<1

\subsection{Quadratic fields inside quaternion algebras}%<<<2
\label{ss:quad-fields}

Let~$B$ be a quaternion algebra over~$ℚ$.
For any quadratic field~$K ⊂ B$ with non-trivial automorphism~$σ$,
we know (by Skolem-Noether) that there exists an element~$j ∈ B ∖ 0$
such that, for all~$x ∈ K$, $j x = σ(x) j$, and~$j^2 = β ∈ ℚ$.
(Moreover, $j$~is determined up to multiplication by~$K^{×}$).
This gives the following map~$B ↪ K^{2×2}$:
$x ∈ K ↦ \mat{x&0\\0&σ(x)}$, $j ↦ \mat{0&j^2\\1&0}$.
This implies that~$x + j y ∈ B ↦ \mat{x&j^2 σ(y)\\y&σ(x)}$
and we easily check that this is an algebra homomorphism.
This map extends to a splitting~$B ⊗_{ℚ} K ≃ K^{2×2}$.

\begin{prop}\label{prop:B-contains-L}%<<<
Let~$L = ℚ[√D]$ be a quadratic extension of~$ℚ$
and~$B / ℚ$ be a quaternion algebra such that~$B ⊂ L^{2×2}$.
Then $B$~contains a sub-field isomorphic to~$L$.
\end{prop}


\begin{proof}
Let~$\acco{i, j}$ be a quaternionic basis of~$B$ over~$ℚ$:
that is, $i^2 = c, j^2 = d ∈ ℚ$,
and~$\Tr i = \Tr j = \Tr i j = 0$.
Since $L^{2 × 2}$~is split over~$L$, it is isomorphic to~$(1,c/L)$,
and has therefore a quaternionic basis~$\acco{i, ε}$ with~$ε^2 = 1$.
Since $\acco{i, j}$~is another quaternionic basis of~$L$,
we have~$j ∈ L[i] · ε$, or~$j = a ε$ with~$a ∈ L[i]$.
Moreover, we see that~$d = j^2 = a ε a ε = a ‾{a} ε^2
 = N_{L[i]/L} (a) ∈ ℚ$.

We now prove the following lemma: let~$z ∈ L[i]$
such that~$N_{L[i]/L} (z) ∈ ℚ$.
Then $z ∈ ℚ[i]^{×} · ℚ[i√D]^{×}$.
We write~$z = x + y √D$ with~$x, y ∈ ℚ[i]$.
Since~$N_{L[i]/L}(z) = (x +_y √D) (‾{x} + ‾{y} √D) ∈ ℚ$,
we see that~$(x ‾y + ‾x y) = 0$.
This means that~$y/x ∈ i ℚ$, or that~$y = i t x$ with~$t ∈ ℚ$.
We then have~$z = x + y √D = x (1 + i √D\, t)$ as required.

Applying this lemma to~$a$, we see that
we may write~$u j = (p + i √D\, q) ε$ with~$u ∈ ℚ[i]^{×}$,
which means that~$(uj)^2 = (p^2 -c D q^2)$.
Consequently:
\begin{equation}
B ≃ \qalg{c,\,p^2 - c D q^2}{ℚ}
  ≃ \qalg{p^2 c, c^2 q^2 D - p^2 c}{ℚ}
  ≃ \qalg{\frac{p^2}{c q^2}, D - \frac{p^2}{c q^2}}{ℚ}.
\end{equation}
In this last basis, we then have~$i^2 + j^2 = D$,
so that~$ℚ[i + j] ≃ L ⊂ B$ as required.
\end{proof}%>>>

\subsection{Complex multiplication points}%<<<2

Let $B$~be an indefinite quaternion algebra over~$ℚ$.
We fix a real quadratic~$K ⊃ ℚ$
and choose one of the two embeddings~$K ⊂ ℝ$.
The construction of~\ref{ss:quad-fields} then defines
an unique map~$η: B → ℝ^{2×2}$.
(Note that the image of~$j K$ is well-defined!).
Let also~$\ro O$ be an order of~$B$.

For any~$z ∈ ℂ$, we write~$Λ(z)$
for the lattice~$η(\ro O) · \mat{z\\1}$ of~$ℂ^2$.
Let~$A(z)$ be the polarized abelian surface~$ℂ^2/Λ(z)$.

We say that $z$~has \emph{complex multiplication} by~$L ⊂ B$
if it is the fixed point of~$η(L)$,
or equivalently if $η(L) · Λ(z) = Λ(z)$.

\begin{thm}
Let~$z ∈ ℂ$. The following are equivalent.
\begin{enumerate}
\item The point $z$~has complex multiplication
(by an imaginary quadratic field~$L$).
\item The abelian surface~$A(z)$ is isogenous to the square of
an elliptic curve~$E$, having complex multiplication (by~$L$).
\item The ring~$\End_{ℚ} (A)$ is isomorphic to~$L^{2×2}$.
\item The ring of QM-automorphisms~$\End_{B} (A)$
is not reduced to~$ℚ$.
\end{enumerate}
\end{thm}


\begin{proof}
(ii) $⇒ $ (iii). If $A(z) ∼ E × E$ then $\End_{ℚ} (A(z)) ≃
\End_{ℚ}^{2×2}$.

(iii) $⇒ $ (ii). By Falting's proof of the Tate conjecture for abelian
varieties over number fields, we know that, for $ℓ$~prime,
\[ \Hom_{ℚ} (A, E × E)⊗ ℤ_{ℓ} \quad ≃ \quad
  \Hom_{\Gal} (T_{ℓ} (A), T_{ℓ} (E) × T_{ℓ} (E)). \]
The assumption~(iii) means that the right-hand side contains an
isomorphism~$ι$. The image of~$ι$ on the left-hand side is an isogeny.

(iii) $⇒ $ (ii), elementary proof.
Since $\End_{ℚ} (A)$~is not a division algebra, $A$~is not simple.
This means that there exists an isogeny~$A ∼ E_1 × E_2$,
where $E_1, E_2$~are elliptic curves.
If~$E_1 ≁ E_2$ then $\End_{ℚ} A ≃ \End_{ℚ} E_1 × \End_{ℚ} E_2$,
which is at most the product of two quadratic fields
and therefore does not contain the quaternion algebra~$B$.
This proves that $E_1$~is isogenous to~$E_2$, so that $A ∼ E_1^2$.
% Assume that $A$~is simple: then $D =
% \End_{ℚ} A$ is a division algebra. Since this acts (freely) on~$T_{ℓ}(A)$,
% which has dimension~$4$, we know that~$\dim_{ℚ} D ≤ 4$.


(iv) $⇒$ (i).
Let~$L = \End_{B} A(z)$ and assume that~$L ≠ ℚ$.
For any~$λ ∈ L ∖ ℚ$, the multiplication-by-~$λ$ map
defines a map~$m_{λ}: ℂ^2 → ℂ^2$, stabilizing~$Λ(z)$
(by the universal property of the universal cover of~$A(z)$).
By the usual properties of abelian varieties,
$m_{λ}$~is a $ℂ$-linear map.
Since~$m_{λ}(z) ∈ η(\ro O)·z$,
there exists~$c ∈ \ro O$ such that~$m_{λ}(z) = η(c) z$.
Moreover, since $λ$~is a $B$-endomorphism,
$m_{λ}$~commutes with all elements of~$η(\ro O)$,
which implies that $c$~lies in the center of~$B$.
Since $B$~is a central simple $ℚ$-algebra, this means that $c ∈ ℚ$.
In other words, $z$~is fixed by the homographic action of~$λ$.
We just showed that $z$~has complex multiplication by~$L$.

(i) $⇒$ (iv), not-working proof.
We write~$X = η(\ro O^{×+}) ∖ \ro H$ for the Shimura curve
and $\ro A → X$ for the relative abelian surface
with quaternionic multiplication by~$\ro O$.

Let~$ι: \acco{z} ↪ X$ be the injection of the point~$z$.
We then know that $A(z) = \ro A ×_{X, ι} \acco{z}$
is the fibre at~$z$ of the surface~$\ro A$.

Assume that $z$~has complex multiplication by a ring~$R$.
This means that there exists~$λ ∈ H ∖ ℚ$
such that~$η(λ)·z = z$. Write~$γ = η(λ) ∈ ℝ^{2×2}$;
then $γ ∘ ι = ι$,
Let~$\ro A_{γ} = \ro A ×_{X} γ$ be the pull-back of~$\ro A$ along~$γ$,
and~$A_{γ} = A(z) ×_{\ro A} \ro A_{γ}$.
Since~$γ ∘ ι = ι$, $A_{γ}$~is the fibre of~$\ro A_{γ}$ above~$z$,
and therefore isogenous (as a $B$-QM surface) to~$A(z)$.

Therefore, the scalar~$λ ∈ H ∖ ℚ$ defines
an endomorphism~$m_{λ}$ of~$A(z)$.
We see that $m_{λ}$~has the same characteristic polynomial as~$λ$,
which means that $m_{λ}$~is an embedding of~$R$ in~$B$.

(i) $⇒ $(iv).
Assume that $z$~has complex multiplication by an element~$x ∈ B ∖ ℚ$.
Since $\Imag z > 0$, $L = ℚ(x)$~is imaginary quadratic over~$ℚ$.
Write~$η(x) = \mat{a&b\\c&d}$.
Then~$x · \mat{z\\1} = (c z + d) \mat{z\\1}$,
so that~$u = (c z + d)$~is an endomorphism of~$Λ(z)$.
Moreover, since $u$~is a homothety, it commutes to~$B$,
so that it is a $B$-endomorphism of~$A(z)$.
Finally, since $L$~is imaginary, $L ≠ K$, therefore~$c ≠ 0$ and~$u ∉ ℝ$.

(iii) $⇒$ (iv).
By Prop.~\ref{prop:B-contains-L}, since $B ⊂ L^{2×2}$,
$B$~contains a sub-field~$L'$ isomorphic to~$L$.
Write~$L' = ℚ[√{D}]$: then the element~$√{D} ∈ B$
is diagonalizable over~$ℚ$,
and therefore of the form~$\mat{√D\\&-√D}$ in some basis of~$L^2$.
This shows that there exists maps~$L ⊂ B ⊂ L^{2×2}$
such that the composition is the map~$x ↦ \mat{x\\&σ(x)}$,
where $σ$~is the non-trivial automorphism of~$L/ℚ$.
We now see that the $L$-homothety matrices
commute with all elements of~$B$, so that~$\End_{B} A = L$.

(iv) $⇒$ (iii)
Let~$R = \End_{ℚ} A ⊃ B$.
Then $C = \End_{B} A$~is the commutant of~$B$ in~$R$.
Since $B$~is central simple, if $R = B$ then~$C = ℚ$,
which is impossible.
Hence~$R ≠ B$.
\end{proof}


Let~$z ∈ X(\ro O)$ be a CM point by
the imaginary quadratic field~$L ⊂ B$.
We say that $z$~has \emph{complex multiplication by~$\ro A = L ∩ \ro O$}.
For any quadratic order~$\ro A$ over~$ℤ$,
we write~$\CM (\ro O, \ro A)$ for the set of points of~$X(\ro O)$
having complex multiplication by~$\ro A$.


\begin{prop}\label{prop:elliptic-cm}
A point~$z ∈ X(\ro O)$ is elliptic
iff it has complex multiplication by a imaginary quadratic order
isomorphic to one of the two quadratic orders
$ℤ[√{-1}]$~or~$ℤ[\frac{1+√{-3}}{2}]$.
\end{prop}

\begin{proof}
The elements~$γ ∈ \ro O$ fixing~$z ∈ \ro H/\ro O$
are entire over~$ℤ$ and of finite order,
and therefore of order~$2$, $3$, $4$~or~$6$ in an imaginary quadratic field.
\end{proof}

\begin{prop}\label{prop:cm-equivalent}
Let~$\ro A, \ro A' ⊂ \ro O$ be two imaginary quadratic orders.
The CM points associated with~$\ro A$ and~$\ro A'$ coincide
iff $\ro A'$~is conjugated to~$\ro A$
by an inner automorphism of~$\ro O$: $\ro A' = x^{-1} \ro A x$
for~$x ∈ \ro O^{×+}$.
\end{prop}

\begin{proof}
Assume~$\ro A' = x^{-1} \ro A x$.
Let~$z$ be a fixed point of~$\ro A$: $η(a) z = z$ for~$a ∈ A$.
Then, for~$a' = x^{-1} a x ∈ \ro A'$, $η(a') (η(x^{-1}) z) = η(x^{-1}) z$,
so that $x^{-1} z$~has complex multiplication by~$\ro A'$.

Conversely, assume that two quadratic orders~$\ro A$, $\ro A'$
have conjugate fixed points~$z, z' = σ z$.
Replacing~$\ro A'$ by~$σ \ro A' σ^{-1}$,
we may assume that~$z = z'$.
We then use Prop.~\ref{prop:fixed-points} to conclude.
\end{proof}

\subsubsection{Examples.}
Let~$B_6$ be the quaternion algebra over~$ℚ$ ramified at
the primes~$2$ and~$3$: for example, $B_6 = \qalg{2,3}{ℚ}$.
Let~$i, j ∈ B_6$ such that~$i^2 = 2$, $j^2 = 3$, $ij + ji = 0$.
A maximal order of~$B_6$ is
$\ro O = \chev {1, i, \frac{1+i+j}{2}, \frac{j+ij}{2}}$.
We fix the real quadratic field~$K = ℚ(√2) ⊂ B_6$
which gives the embedding
\begin{equation}
η: B_6 → ℝ^{2×2}, i ↦ \mat{√2\\&-√2},
j ↦ \mat{&3\\1}, ij ↦ \mat{&-3√2\\√2}.
\end{equation}

Let~$α = \frac{i+3ij}{2}$; then we check that~$α^2 = -13$,
so that~$ℚ(α) ≃ ℚ(√{-13}) ⊂ B_6$.
We have~$η(α) = \frac{√2}{2} \mat{1&-9\\3&-1}$,
so that the fixed point of~$ℚ(α)$
is the image in~$X(\ro O)$ of~$z(α) = \frac{1+√{-26}}{3}$.

Let~$β = \frac{i+ij}{2}$; then~$β^2 = -1$,
so that~$ℚ(β) = ℚ(√{-1}) ⊂ B_6$.
We have~$η(β) = \frac{√2}{2} \mat{1&-3\\1&-1}$,
so that the fixed point of~$ℚ(β)$
is the image in~$X(\ro O)$ of~$z(β) = 1 + √{-2}$.

\paragraph{Unramified case.} Let~$B_1 = (1,1 / ℚ) = ℚ^{2×2}$
be the split quaternion algebra over~$ℚ$.
We write $i = \mat{1\\&-1}$, $j = \mat{&1\\1}$, $ij = \mat{&1\\-1}$,
so that~$i^2 = j^2 = 1$ and~$(ij)^2 = -1$.
Let~$\ro O(N) = \chev{1, \frac{1+i}{2}, \frac{N+1}{2} j,
  \frac{j+ij}{2}}$.
We check that $\ro O(N)$~is an order of~$B_1$.
Its image~$η(\ro O(N))$
is the congruence group~$Γ_0(N) = \acco{\mat{a&b\\c&d} ∈
\mathrm{SL}_2(ℤ),
  c ≡ 0 \pmod{N}}$.
Therefore, the Shimura curve~$X(\ro O(N))$
is the classical modular curve~$X_0(N)$.

Let~$d ∈ ℤ$ and~$δ = \frac{d+1}{2} i + \frac{d-1}{2} ij ∈ \ro O$.
The fixed point of~$η(δ) = \mat{&d\\1}$ in~$\ro H$
is~$z = √{d}$, which is imaginary if~$d < 0$.

\section{In characteristic~$p$: supersingular points}%<<<1

% \subsection{Reminders about Honda-Tate theory}
% 
% We give a short reminder about classification of isogeny classes of
% abelian surfaces over a finite field~$k = \F_q$.
% \begin{equation}
% \frac{\acco{\text{abelian varieties over~$k$}}}{\text{isogeny}}
%   \;≃\;
% \frac{\acco{\text{$q$-Weil numbers}}}{\text{Galois}}
% \end{equation}
% where a \emph{$q$-Weil number} is an algebraic integer~$α$
% such that, for all~$σ: ℚ(α) ↪ ℂ$, $\abs{σ(α)} = √q$.
% 

Let~$A$ be an abelian surface defined over the field~$k$,
with quaternionic multiplication by the indefinite algebra~$B$,
\emph{i.e.} equipped with an (injective) morphism~$B ↪ R = \End A ⊗ ℚ$.

\begin{thm}
Let~$A$ be an abelian surface over~$k$,
with QM by the indefinite quaternion algebra~$B$.
Then either
\begin{enumerate}
\item $A$~is isogenous to the square $E^2$ of an elliptic curve, or
\item $A$~is simple and $\End_{ℚ} A = B$.
\end{enumerate}
\end{thm}


\paragraph{If $A$~is not simple,}
then $A$~is isogenous to a product~$E_1 × E_2$ of two elliptic curves.
If $E_1 ≁ E_2$ then since~$R = \End_{ℚ} E_1 × \End_{ℚ} E_2$,
we have at least one injection~$B ↪ \End_{ℚ} E_i$,
so that the curve~$E_i$ is supersingular.
However, in this case, the endomorphism ring of~$E_1$ is
the quaternion algebra~$B_{p,∞}$ ramified at~$\acco{p,∞}$.
Since $B_{p,∞}$~is a definite quaternion algebra, we have~$B ≠ B_{p,∞}$,
which is impossible. This proves that~$E_1 ∼ E_2$.

We therefore have~$A ∼ E^2$ and~$R = \End_{ℚ} A = (\End_{ℚ} E)^{2×2}$.
Let~$C = \End_{ℚ} E$.
If $C = ℚ$ then $R = ℚ^{2×2}$ is a (split) quaternion algebra over~$ℚ$
and there exists a map~$B → R$ iff $B = R$.
If $C$~is an imaginary quadratic field then it must split~$B$.
The last case is when $C$~is the quaternion algebra~$B_{p,∞}$.
We can show that,
for any indefinite quaternion algebra~$B$ and any prime~$p$,
there exists an embedding~$B ↪ (B_{p,∞})^{2×2}$.

\paragraph{If $A$~is simple,}
then its endomorphism algebra~$R = \End_{ℚ} A$ is a simple algebra.
Let~$K$ be the center of~$R$.
Since $\dim A = 2$, the field~$K$ is an extension of~$ℚ$
of degree~$1$, $2$ or~$4$.

If $[K:ℚ] = 4$ then~$R = K$ and $R$~is commutative,
which is impossible since~$B ⊂ R$.

If $K = ℚ$ then, since $R$~is central simple over~$ℚ$,
it is a quaternion algebra over~$ℚ$, hence~$R = B$.

If $K$~is a real quadratic field, then $R$~is
a quaternion algebra over~$K$, containing~$B$ and therefore~$B ⊗ K$.
Since $A$~is simple, $R$~is not split over~$K$.
Therefore, $K$~does not split~$B$,
and $R$~contains a real quadratic extension~$K'$ of~$K$,
which is therefore a totally real quartic extension of~$ℚ$.
By [Mumford, Corollary p. 191], this implies that~$4 ∣ \dim A$,
which is impossible.

Assume that $K$~is an imaginary quadratic field.
Then since $R$~is a quaternion algebra over~$K$
containing~$B$, we can show that~$R = B ⊗_{ℚ} K$.

We can show that this last case may only happen when
the base field~$k$ has characteristic~$p > 0$.
$\End_{ℚ} A$~contains a CM quartic field~$L$.
If~$p = 0$ then $A$~would have its endomorphism ring equal to
the CM field~$L$; this impossible since $\End_{ℚ} A$~is not commutative.

Let~$\fr q$ be a place of~$K$ that does \emph{not} divide~$p$.
\textsf{XXX} (by Honda-Tate?) Then $\fr q$~is split in~$R$:
$R⊗_{K} K_{\fr q} ≃ K_{\fr q}^{2×2}$.
Since $R$~is a division algebra,
it is not split at all places of~$K$,
and is therefore ramified at the two places~$\fr p, \fr p'$~dividing~$p$.
This means that the discriminant of~$R$ over~$K$ is~$\fr p \fr p' = p$.

Assume that $p$~does not divide the discriminant of~$B/ℚ$.
Then the embedding $B ↪ R$, when tensoring by~$ℚ_p$,
gives an embedding
\begin{equation}
B ⊗_{ℚ} ℚ_{p} = ℚ_p^{2×2} \quad↪\quad
R ⊗_{ℚ} ℚ_p = R ⊗_{K} (K_{\fr p} ⊕ K_{\fr p'}).
\end{equation}
Since the algebra~$ℚ_p^{2×2}$ has nilpotent elements
while~$R_{\fr p} ⊕ R_{\fr p'}$ does not, this is a contradiction.

Therefore, $p$~divides $\mathrm{disc} B/ℚ$.
This means that $B⊗_{ℚ} ℚ_p$~is a division algebra.
The Tate module~$T_p(A)$ has dimension~$0$, $1$~or~$2$.
The map~$B ↪ \End_{ℚ} (A)$ then gives a map
$ρ: B ⊗ {ℚ_p} → \End_{ℚ_p} (T_p(A) ⊗_{ℤ_p} ℚ_p)$.
If $T_p(A) ≠ 0$, then $ρ(1) = 1$ and $ρ$~is therefore injective.
This gives an embedding~$B_p ↪ ℚ_p^{i×i}$ for~$i ≤ 2$,
which is impossible.


\end{document}
